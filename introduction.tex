\chapter{Introduction}
In the past year literature has seen significant efforts in the automated collection, integration, and analysis of the vast amount of heterogeneous biomedical information published online. This information can be semi-automatically mined from various knowledge bases, harmonized and standardized, and condensed into integrated biomedical information structures known as knowledge graphs (KGs). In these graphs, information is represented by entities or concepts (such as proteins, drugs, and chemical compounds), which are nodes linked by edges representing known relationships.

Prominent examples of biomedical knowledge graphs in the past five years include those used for COVID-19 research~\cite{ReeseJustinT.2021KAFt}, the discovery of key exposure factors for female reproductive disorders~\cite{ChanLaurenE2024Pnae}, and investigations into RNA molecules to infer new RNA targets for personalized treatments\cite{CavalleriEmanuele2024Aokg}. To extract meaningful knowledge from these graphs, graph representation learning techniques have been proposed\cite{Hamilton2020GraphRL}, \cite{li2022graphrepresentationlearningbiomedicine}. These techniques compute representative embeddings for graph elements and use these embeddings to predict unknown categories of nodes or edges or the existence of edges between nodes.

Despite the progress in data integration research, a clear divide remains between two predictive strategies, as pointed out in the vision of individualized knowledge graphs\cite{PingPeipei2017IKGA}. On one side are predictive methods that work on datasets representing specific cases, which often yield knowledge that is not generalizable enough to be applied to compute reliable predictions on novel cases. On the other side are methods that work on KGs to uncover broad, static knowledge that is often too general for practical application. What is missing is an approach that enables predictions on specific sample data using the broad knowledge from KGs, while also deriving new relationships within a KG based on information from specific sample data. To address this issue, the first challenge is to connect these two knowledge sources. This requires an investigation into how nodes representing specific patients or samples can be linked to nodes representing broader concepts, such as genes or specific RNA molecules. Secondly, given the vast number of nodes in a KG compared to the often limited number of cases in medical studies, new techniques should be developed to process a KG in a way that biases its representation to retain information from less represented nodes.

In \Cref{kgs} and \Cref{grl} we present the theoretical background and applications of knowledge graphs and graph representation learning.