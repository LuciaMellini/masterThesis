\begin{figure}
    \begin{subfigure}{0.48\textwidth}
        \centering
        \begin{tikzpicture}[scale=1.7]
            \draw (0,0) circle (1);
            \clip (0,0) circle (1);
            \hgline{30}{-30}{black!70}
            \hgline{180}{270}{blue!20}
            \hgline{30}{120}{black!70}
            \hgline{0}{180}{red!50}
        \end{tikzpicture}
        \caption{Poincaré ball model.}
        \label{fig:poincareBall}
    \end{subfigure}%
    \hfill
    \begin{subfigure}{0.48\textwidth}
        \centering
        \begin{tikzpicture}
        \begin{axis}[
            view={30}{10},
            colormap name=custom,
            shader=interp,
            axis lines=none,
            xmin=-3, xmax=3,
            ymin=-3, ymax=3,
            zmin=0, zmax=5,
        ]
        \addplot3[
            surf,
            domain=-2.8:2.8,
            y domain=-2.8:2.8,
            samples=20,
            z buffer=sort,
            opacity=0.7
        ] {sqrt(x^2 + y^2 + 1)};
        \end{axis}
        \end{tikzpicture}   
        \caption{Hyperboloid model.}
        \label{fig:hyperboloid}
    \end{subfigure}
    \caption{Models of hyperbolic space.}
    \label{fig:hyperbolicmodels}
\end{figure}
