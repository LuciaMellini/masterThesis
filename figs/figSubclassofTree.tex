\begin{figure}
    \centering
        \begin{tikzpicture}
        [edge from parent fork down,
        sibling distance=40mm,
        every node/.style={anchor=west, align=center}]
        \small
        \node {disease } 
        [sibling distance=30mm]
        child {node {human disease}
            [sibling distance=35mm] 
            child {node {digestive system disorder}
                [sibling distance=32mm]
                child {node {intestinal disorder}
                    [sibling distance=37mm]
                    child {node {intestinal obstruction}
                        [sibling distance=15mm]
                        child {node {ileus}
                            [sibling distance=27mm]
                            child {node {meconium ileus}    
                                [sibling distance=65mm]                 
                                    child {node {\small \textbf{cystic fibrosis associated meconium ileus}}}                                        
                                    child {node {\dots}}                    }
                            child {node {\dots}}
                       }
                        child {node {\dots}}
                   }
                    child {node {\dots}}
               }
                child {node {\dots}}
           }
            child {node {\dots}}
       }
        child {node {perinatal disease}
            [sibling distance=0mm]
            child {node {\small \textbf{cystic fibrosis associated meconium ileus}}}
       }
        child {node {\dots}};
    \end{tikzpicture}
    \caption{A subtree composed of disease type nodes linked by \emph{subclassof} edges centered on the leaf disease \emph{cystic fibrosis associated meconium ileus} .}
    \label{fig:SubclassofTree}
\end{figure}