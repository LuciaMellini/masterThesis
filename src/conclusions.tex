\chapter{Conclusions and future work}
We can deduce from the results shown in the previous chapter that a model trained using the HGCN approach is not adequate for the task of differential diagnosis on the PatientKG graph. In particular, the \emph{Has disease} connections have not been identified correctly for novel pairs of patients and diseases, as shown by the low values of ROC-AUC and average precision metrics on the test set. This could be due to a few reasons. 

As already stated, the HGCN model strongly hones in on the hierarchical structure of the graph. PatientKG as a whole can be seen as a collection of multiple tree-like subgraphs connected to each other by non-hierarchical relations, as discussed in \Cref{sec:underlyingKG}. So, the results suggest that possibly a more elaborate model is necessary. For example, a graph neural network that combines hyperbolic and Euclidean learning, such as the one proposed in \cite{Chami2021representationLearningAlgorithmsHyperbolicSpaces}, could be more effective. 

Another reason could be the extreme class imbalance of the dataset, which makes it difficult for the model to learn to identify positive edges. A possible solution could be to experiment with different loss functions that are more robust to class imbalance. Also, one could experiment with a targeted negative sampling strategy; for simplicity we have opted for negative samples chosen at random in the graph. It would be interesting to guide the learning process by proposing negative samples in the neighbourhood of the positive ones, or by levaraging some similarity notion among edges in the graph, or alternatively by using domain knowledge to select challenging negative samples.

To counteract the small amount of connections between patients and diseases with respect to the underlying knowledge graph, one could influence the graph encoding method by trying to force the model to focus more on the edges connected to the patients. This could be done by weighting the loss function to stress the correct prediction of edges involving patients more, or by oversampling batches with a higher number of patients.