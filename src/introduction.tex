\chapter{Introduction}

In the past years literature has seen significant efforts into gathering biomedical knowledge in the form of knowledge graphs. Numerous graph representation learning techniques have been proposed to extract meaningful knowledge from these graphs. Despite the progress in data integration research, a clear divide remains between two predictive strategies, as pointed out in the vision of individualized knowledge graphs\cite{PingPeipei2017IKGA}. On one side are predictive methods that work on datasets representing specific patient cases, which often yield knowledge that is not generalizable enough to be applied to compute reliable predictions on novel cases. On the other side are methods that work on KGs to uncover broad, static knowledge that is often too general for practical application. What is missing is an approach that enables predictions on specific sample data using the broad knowledge from KGs, while also deriving new relationships within a KG based on information from specific sample data. To address this issue, the first challenge is to connect these two knowledge sources. This requires an investigation into how nodes representing specific patients or samples can be linked to nodes representing broader concepts, such as genes or specific RNA molecules. Secondly, given the vast number of nodes in a KG compared to the often limited number of cases in medical studies, new techniques should be developed to process a KG in a way that biases its representation to retain information from less represented nodes.

The techniques analyzed in this work are addressed at the task of differential diagnosis of patients suffering from Mendelian diseases. The goal is to predict the most likely rare disease for a patient based on their symptoms and genetic information. At this scope we have prepared a cured KG integrated with patient data. To extract meaningful predictions based on this KG we have explored hyperbolic graph representation learning techniques. We have based the choice on the intuition that ontological data present in the KG is organized hierarchically, so it would be beneficial to stress these underlying tree-like structures.

\newpage
\section*{Structure of the thesis}
The work is organized as follows:
\begin{description}
    \item[\Cref{kgs}] gives an overview of knowledge graphs: their structure, how to deduce and induce information from them. Seen our scope we then focus on knowledge graphs used in the biomedical domain.
    \item[\Cref{grl}] introduces graph representation learning, a technique to learn succinct descriptions of graphs. We go through the evolution of this approach to justify the methods currently used. In \Cref{sec:embeddingsKGs} we delve into how graph representation learning has been adapted for knowledge graphs.
    \item[\Cref{hyperbolic}] highlights some geometry concepts to understand the intuition behind hyperbolic geometry, its connection to tree metric spaces and the tools to build learning models that operate in hyperbolic spaces. 
    \item[\Cref{hrl}] explores hyperbolic representation learning methods for hierarchical graphs, with a focus on the HGCN technique.
\end{description}