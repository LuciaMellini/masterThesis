\chapter{Hyperbolic geometry}\label{hyperbolic}
We introduce hyperbolic spaces by putting them in contrast with the Euclidean space we are more accustomed to. We then briefly discuss (Reimanninan) manifolds, which are the mathematical framework for hyperbolic spaces. In addition, using the lens of manifolds, we highlight some of characteristics of hyperbolic spaces. Lastly, leading up to \Cref{hrl}, we stress the connection between hyperbolic spaces and tree-like structures.

\section{Beyond Euclidean geometry}

\subsection{The parallel postulate}\label{sec:parallelPostulate}
In 300 BC Euclid wrote his \textit{Elements}, consisting of thirteen books (translated in~\cite{Fitzpatrick2008euclidElementsGeometry}). In the first book he spoke about five postulates which form the basis of Euclidean geometry. A literal translation of these the postulates (with the exception of the fifth) is:
\begin{enumerate}
    \item Let it have been postulated to draw a straight line from any point to any point.
    \item To produce a finite straight line continuously in a straight line.
    \item To describe a circle with any center and distance.
    \item That all right angles are equal to another.
    \item (Parallel Postulate) Through a point $P$ outside of an infinitely long line $\ell$ there is only one infinitely long line that does not cut the first line $\ell$.
\end{enumerate}

This Parallel Postulate is known as Playfair’s postulate, the most used alternative for Euclid’s real fifth postulate. The original fifth postulate states:
\begin{quote}
    That, if a straight line on two (other) straight lines makes the interior angles on the same side less than two right angles, the two straight lines, if produced indefinitely, meet on that side on which the angles are less than the two right angles.
\end{quote}


It is possible to obtain internally consistent models of geometries which obey to all postulates except for the parallel postulate. These geometries are denoted as \term{non-Euclidean geometries}. One way to do this might be to insist that any two lines intersect. This gives rise to so called \term{Elliptic Geometry}. Another way of obtaining a non-Euclidean geometry is to state that the line through point $P$ that does not meet $\ell$ is not unique. This determines \term{Hyperbolic Geometry}, in which the parallel postulate is replaced by the following.


\begin{postulate}(Hyperbolic parallel postulate)
    Given any line $\ell$ and a point $P$ not on $\ell$, there are at least two lines though $P$ that do not meet $\ell$. In other words there are at least two lines through $P$ that are parallel to $\ell$.
\end{postulate}


\begin{figure}
    \centering
    \begin{subfigure}{0.25\textwidth}
        \centering
        \begin{tikzpicture}[scale=0.6]
            \begin{axis}[
                    view={30}{30},
                    axis lines=none,
                    colormap name = custom,
                    samples=50,
                    domain=-1:1,
                    y domain=-2*pi:2*pi
            ]
            \addplot3[
                surf,
                shader=interp,
                    z buffer=sort
            ] ({0.2+cosh(x)*cos(deg(y))},
            {0.2+cosh(x)*sin(deg(y))},
            {sinh(x)});     
            \end{axis}
        \end{tikzpicture}
        \caption{Negative curvature}
     \end{subfigure}
     \hfill
     \begin{subfigure}{0.25\textwidth}
        \centering
        \begin{tikzpicture}[scale=0.6]
            \begin{axis}[
                    view={30}{30},
                    axis lines=none,
                    colormap name=custom,
                    samples=50,
                    domain=0:2*pi,
                    y domain=-2*pi:2*pi
            ]
            \addplot3[
                    surf,
                    shader=interp,
                    z buffer=sort
            ] ({cos(deg(x))},
            {sin(deg(x))},
            {y});
            \end{axis}
        \end{tikzpicture}
        \caption{Zero curvature}
     \end{subfigure}
     \hfill
     \begin{subfigure}{0.25\textwidth}
        \centering
        \begin{tikzpicture}[scale=0.6]
            \begin{axis}[
                    view={30}{0},
                    axis lines=none,
                    colormap name=custom,
                    samples=50,
                    domain=-90:90,
                    y domain=0:360
            ]
            \addplot3[
                    surf,
                    shader=interp,
                    z buffer=sort
            ]
            ({cos(x)*cos(y)},
            {cos(x)*sin(y)},
            {sin(x)});
            \end{axis}
        \end{tikzpicture}
        \caption{Positive curvature}    
    \end{subfigure}
    \caption{Surfaces with various curvatures.}
    \label{fig:spaceCurvatures}
\end{figure}
Spaces in non-Euclidean geometries can be classified according to their curvature, which measures the deviation from flat Euclidean spaces. Specifically, spaces in elliptic geometry have positive curvature, whilst those in hyperbolic geometry possess negative curvature, as depicted in \Cref{fig:spaceCurvatures}. 

\subsection{Reimannian manifolds}
A Riemannian manifold is a mathematical space that generalizes the concept of curved surfaces and higher-dimensional spaces, allowing us to define distances, angles, and curvature in a smooth way. 


\begin{definition} (Reimannian manifold)
Given a smooth manifold $M$, a Riemannian metric is a smooth function that assigns to each $p \in M$ an inner product on the tangent space $T_pM$:
\begin{equation*}
    g_p: T_pM \times T_pM \to \R.
\end{equation*}
    A Reimannian manifold is a pair $(M,g)$ where $M$ is a smooth manifold and $g$ is a Riemannian metric.
\end{definition}

Simply put, a \term{manifold} is a topological space\footnote{A \term{topological space} is a set of points, along with a topology, an additional structure which can be defined as a set of neighbourhoods for each point that satisfy some axioms formalizing the concept of closeness.} that locally resembles Euclidean space near each point. More precisely, a $d$-manifold  is a topological space with the property that each point has a neighborhood that resembles an open subset of $d$-dimensional Euclidean space. 

A \term{smooth manifold} is such when functions and coordinate changes along the manifold are infinitely differentiable. The associated metric entails that at each point $p$ lengths and angles are measured differently.

For more details on foundations of manifolds refer to the Appendix in~\cite{Chami2021representationLearningAlgorithmsHyperbolicSpaces}, or~\cite{doCarmo1992riemannianGeometry}\cite{Lee2003smooth}.


\section{Hyperbolic spaces}
Building on the concept of manifolds we analyze how distances and straight lines morph in hyperbolic spaces. In practice it would not be possible to work directly on hyperbolic spaces. Hence, we introduce some ways to operate on hyperbolic spaces and help in their representation.
\begin{description}
    \item[Exponential and logarithmic maps] In \Cref{sec:expLogMaps} we provide a mapping between hyperbolic spaces and their Euclidean tangent planes. Because the tangent space has a linear structure, we will rely on these maps to leverage some of the tools of Euclidean geometry for hyperbolic methods, such as optimization or neural network operations.
    \item[Poicaré ball] There exist five equivalent models of hyperbolic spaces\footnote{The possible models of hyperbolic spaces are the following: the Poincaré ball, the Poincaré half-ball, the Klein disk, the hyperboloid model and the upper half-space model.}. In \Cref{sec:poincareBall} we focus on a model of hyperbolic geometry called the Poincaré ball, that represents the infinite hyperbolic space in a finite ball, and is therefore very useful for visualizations.
\end{description}

More in depth discussion on hyperbolic geometry can be found in~\cite{Anderson2006hyperbolicGeometry}\cite{Ramsay2013introductionHyperbolicGeometry}.

\subsection{Distance function and geodesics}
In negative curvature spaces, the space bends like a saddle and triangle angles sum to less than $\pi$. We show a section of a hyperbolic paraboloid (described by function $z=x^2-y^2$) in \Cref{fig:hyperbolicSpace}.

\begin{figure}
    \centering
    \begin{tikzpicture}
        \begin{axis}[
            view={160}{70},            
            axis lines=none,            
            colormap name = custom,
            xmin=-2, xmax=2,
            ymin=-2, ymax=2,
            zmin=-2, zmax=2,
        ]
        
        % Plot the hyperbolic paraboloid
        \addplot3[
            surf,
            shader=interp,        
            colormap name=custom,
            domain=-1.7:1.7,
            domain y=-2:1.5,
            samples=30,
            opacity=1,
        ] {x^2 - y^2};
        
        % Define vertices of the triangle (lying on the surface)
        \newcommand{\vertexA}{(-1, -1, 0)}    % z = (-1)^2 - (-1)^2 = 0
        \newcommand{\vertexB}{(1, -1, 0)}     % z = 1^2 - (-1)^2 = 0
        \newcommand{\vertexC}{(0, 1, -1)}     % z = 0^2 - 1^2 = -1
        
        % Edge from A to B (along y = -1, z = x^2 - 1)
        \addplot3[
            thick,
            red,
            smooth,
            samples=50,
            samples y=0,
            domain=-1:1,
        ] ( 
            {x},                % x(x) = -1 → 1
            {-1},               % y(x) = -1 (constant)
            {x^2 - 1}         % z(x) = x(t)^2 - y(t)^2 = t^2 - 1
        );
        
        % Edge from A to C (parametric curve)
        \addplot3[
            thick,
            red,
            smooth,
            samples=50,
            samples y=0,
            domain=0:1,
        ] ( 
            {-1 + x},           % x(x) = -1 → 0
           { -1 + 2*x},         % y(x) = -1 → 1
            {(-1 + x)^2 - (-1 + 2*x)^2}  % z(x) = x(t)^2 - y(t)^2
        );
        
        % Edge from B to C (parametric curve)
        \addplot3[
            thick,
            red,
            smooth,
            samples=50,
            samples y=0,
            domain=0:1,
        ] ( 
            {1 - x},            % x(x) = 1 → 0
            {-1 + 2*x},         % y(x) = -1 → 1
            {(1 - x)^2 - (-1 + 2*x)^2}  % z(x) = x(t)^2 - y(t)^2
        );
        \end{axis}
    \end{tikzpicture}
   
    \caption{Section of a hyperbolic space.}
    \label{fig:hyperbolicSpace}
\end{figure}




We first introduce some useful quantities for our discussion.

\begin{definition}(Minkowski inner product) \label{def:minkowskiInnerProduct}
    Given two vectors $x,y \in \R^d$, their Minkowski inner product is defined as:
    \begin{equation*}
        \langle x, y \rangle_\Lcal = -x_0y_0+\sum_{i=1}^d x_iy_i
    \end{equation*}    
\end{definition}

\begin{definition}(Minkowski norm) \label{def:minkowskiNorm}
    Given a vector $x \in \R^d$, its Minkowski norm is defined as:
    \begin{equation*}
        \|x\|_\Lcal = \sqrt{\langle x, x \rangle_\Lcal}.
    \end{equation*}
\end{definition}

For each point $p$ in a hyperbolic space $\Hy^d$ the tangent space at $p$---$T_p\Hy^d$---is a $d$-dimensional vector space containing all possible directions of paths leaving from $p$. Formally the tangent space to point $p$ is described as:
\begin{align*}
    T_p\Hy^d 
    &= \left\{u\in \R^{d+1} : \langle u, p \rangle_\Lcal = 0\right\} \\
    &= \left\{(v_0,v) \in \R^{d+1} : v_0 = \frac{v \cdot p}{\sqrt{1 + \|p\|^2_2}}\right\}
\end{align*}

An important concept in Euclidean geometry, arising from Euclid's first postulate (\Cref{sec:parallelPostulate}), is that of straight lines, which are shortest paths between points. We can generalize this concept to manifolds by seeking curves with minimal length, called \term{geodesics}. A geodesic is a curve $\gamma(t)=(\gamma(t)^1, \dots, \gamma(t)^d)$ in a manifold $M$ connecting points $p,q\in M$ with minimum length. 

In Euclidean space geodesics are straight lines in $\R^d$ and for given parameters $a,b\in \R^d$ they have the form 
\begin{equation}
    \gamma(t) = t\cdot a + b
\end{equation}
which . 

In the case of hyperbolic spaces these length-minimizing curves have a different mathematical expression. Let $p\in\Hy^d$ and $v\in T_p\Hy^d$, assuming $\langle v,v\rangle_\Lcal $. The unique geodesic $\gamma_{p,v}(\cdot)$ such that $\gamma_{p,v}(0)=p$ and $\dot{\gamma}_{p,v}(0)=v$ is given by:

\begin{equation*}
    \gamma_{p,v}(t) = \cosh(t)p + \sinh(t)v.
\end{equation*}
The corresponding intrinsic distance\footnote{An \term{intrinsic distance} refers to the shortest path between two points measured \emph{within} a given space or surface, rather than through an external surrounding space.}  between two points $p,q\in\Hy^d$ can be computed as:

\begin{equation*}
    d_\Hy(p, q) = \text{arcosh}(-\langle p, q \rangle_\Lcal).
\end{equation*}

\subsection{Exponential and logarithmic maps}\label{sec:expLogMaps}
We would like to associate the points in a manifold to those on a tangent Euclidean space. The tangent space maps to the manifold via an exponential map, and conversely, a logarithmic map translates a point on the manifold to the tangent space. 

When dealing with hyperbolic spaces, given $p \in \Hy^d$ and a tangent vector $v \in T_p\Hy^d$, the exponential map $\exp_p: T_p\Hy^d \to \Hy^d$ assigns to $v$ the point $\exp_p(v) := \gamma(1)$, where $\gamma$ is the unique geodesic satisfying $\gamma(0) = p$ and $\dot{\gamma}(0) = v$. The logarithmic map reverses back to the tangent space at $p$ such that $\log_p(\exp_p(v)) = v$. In hyperbolic spaces these operations form a bijection between the entire hyperbolic space and the tangent space at a point. The analytic expressions for the exponential and logarithmic maps are given below.

\begin{align*}
    \exp_p(v) &= \cosh(\|v\|_\Lcal)p + \sinh(\|v\|_\Lcal) \frac{v}{\|v\|_\Lcal}\\
    \log_p(y) &= d_{\Lcal}(p,y)\frac{y + \langle p,y \rangle_{\Lcal}p}{\|y + \langle p,y \rangle_{\Lcal}p\|_{\Lcal}}
\end{align*}

\subsection{Poincaré ball model}\label{sec:poincareBall}
Poincaré's representation of hyperbolic geometry is based on the points in the unit ball in $d$ dimensions, formally described as
\begin{equation*}
    \B^d = \{x \in \R^d: \|x\|^2_2 < 1\}.
\end{equation*}
When the unit ball is endowed with the family of inner products
\begin{equation*}
    g_p = \left(\frac{2}{1-\|p\|^2_2}\right)^2\mathbb{I}_d,
\end{equation*}
the pair $(\B^d,g)$ forms a Riemannian manifold of constant negative curvature. The induced distance between two points $p,q$ in $\B^d$ can be computed as
\begin{equation*}
    d_\B(p,q) = \text{arcosh}\left(1 + 2\frac{\|p-q\|^2_2}{(1-\|p\|^2_2)(1-\|q\|^2_2)}\right).
\end{equation*}


\begin{figure}
  \centering
  \includegraphics[width=0.5\textwidth]{figs/hyperboloidToPoincare.jpg}

    % \begin{tikzpicture}
    %   \begin{axis}[
    %     view={30}{10},
    %     colormap name=custom,
    %     shader=interp,
    %     axis lines=none,
    %     xmin=-3, xmax=3,
    %     ymin=-3, ymax=3,
    %     zmin=0, zmax=5,
    % ]
    
    % % Upper sheet of hyperboloid
    % \addplot3[
    %     surf,
    %     domain=-2.8:2.8,
    %     y domain=-2.8:2.8,
    %     samples=20,
    %     z buffer=sort,
    %     opacity=0.7
    % ] {sqrt(x^2 + y^2 + 1)};
    
    % % Geodesic curve (properly bounded)
    % \addplot3[
    %     red,
    %     thick,
    %     samples=20,
    %     samples y=0,
    %     domain=-1.8:2\,  % Match the hyperboloid's x-domain
    % ] (
    %     {x},
    %     {1},
    %     {sqrt(x^2 + 2)}  % Changed from +4 to +1 to stay on the surface
    % );
    
    % \end{axis}
    % \end{tikzpicture}
  \caption{Illustration of the hyperboloid (top) and its connection to the Poincaré ball (bottom)\cite{Chami2021representationLearningAlgorithmsHyperbolicSpaces}.}
  \label{fig:hyperboloidToPoincareBall}
\end{figure}

The Poincaré ball model of hyperbolic space is isomorphic\footnote{An \term{isomorphism} is a bijective mapping between two structures that preserves their operations, relationships and properties.} to the hyperboloid model (\Cref{fig:hyperboloidToPoincareBall}), and the stereographic projection\footnote{The \term{stereographic projection} is a mapping that projects a sphere onto a plane, preserving shapes locally.}
\begin{equation*}
    (x_0, x_1, \ldots, x_n) \mapsto \left(\frac{x_1}{1+x_0}, \ldots, \frac{x_n}{1+x_0}\right)
\end{equation*}
is an isometry between $\Hy^d$ and the Poincaré bell $\B^d$. Its inverse map is given by
\begin{equation*}
    (y_1, \ldots, y_n) \mapsto \left(\frac{1 + \sum_i y_i^2}{1 - \sum_i y_i^2}, \frac{2y_1}{1 - \sum_i y_i^2}, \ldots, \frac{2y_n}{1 - \sum_i y_i^2}\right).
\end{equation*}
This yields geodesics that are either straight lines that go through the origin of the ball, or segments of circles that are perpendicular to the boundary of the ball as shown in \Cref{fig:hyperbolicGeodesics}.

\begin{figure}
    \begin{subfigure}{0.48\textwidth}
        \centering
        \begin{tikzpicture}[scale=1.7]
            \draw (0,0) circle (1);
            \clip (0,0) circle (1);
            \hgline{30}{-30}{black!70}
            \hgline{180}{270}{blue!20}
            \hgline{30}{120}{black!70}
            \hgline{0}{180}{red!50}
        \end{tikzpicture}
        \caption{Poincaré ball model.}
        \label{fig:poincareBall}
    \end{subfigure}%
    \hfill
    \begin{subfigure}{0.48\textwidth}
        \centering
        \begin{tikzpicture}
        \begin{axis}[
            view={30}{10},
            colormap name=custom,
            shader=interp,
            axis lines=none,
            xmin=-3, xmax=3,
            ymin=-3, ymax=3,
            zmin=0, zmax=5,
        ]
        \addplot3[
            surf,
            domain=-2.8:2.8,
            y domain=-2.8:2.8,
            samples=20,
            z buffer=sort,
            opacity=0.7
        ] {sqrt(x^2 + y^2 + 1)};
        \end{axis}
        \end{tikzpicture}   
        \caption{Hyperboloid model.}
        \label{fig:hyperboloid}
    \end{subfigure}
    \caption{Models of hyperbolic space.}
    \label{fig:hyperbolicmodels}
\end{figure}


\section{Connections between hyperbolic spaces and trees}
We now discuss the connections between hyperbolic spaces and tree-like structures, that motivate the use of hyperbolic spaces to represent hierarchical data, as we will discuss in \Cref{hrl}. We explore this relation theoretically by correlating hyperbolic metrics and tree metrics (\Cref{sec:hyperbolicTreeMetrics}), and more intuitively looking at the structure of trees (\Cref{sec:expGrowth}).

\subsection{Hyperbolic and tree metrics}\label{sec:hyperbolicTreeMetrics}
Theoretical studies have shown that many real-world networks exhibit hierarchical structures with underlying hyperbolic geometry~\cite{Krioukov2010HyperbolicGeometryComplexNetworks}\cite{Papadopoulos2012popularityVSSimilarityGrowingNetworks}. To identify these tree-like structures we turn to notion of Gromov $\delta$-hyperbolicity~\cite{gromov1987hyperbolic}\cite{adcock2013tree}\cite{chen2013hyperbolicity}. To define $\delta$-hyperbolic spaces we introduce the Gromov product~\cite{gromov1987hyperbolic}. This quantity measures how close two points are to each other relative to a third point.  

\begin{definition}[Gromov product]
    In any metric space $(X,d)$, the Gromov product of two points $x,y\in X$ with respect to a third point $z\in X$ is defined as:
    \begin{equation*}
        \langle x,y \rangle_z = \frac{1}{2}\left(d(x,z) + d(y,z) - d(x,y)\right).
    \end{equation*}
\end{definition}

\begin{definition}[$\delta$-hyperbolicity, four-point condition]
    A metric space $(X,d)$ is $\delta$-hyperbolic if there exists $\delta\geq0$ such that for all $x,y,z,w\in X$
    \begin{equation*}
        \langle x,y\rangle_z \geq \min\{\langle x,w\rangle_z, \langle y, w\rangle_z\} - \delta.
    \end{equation*}
\end{definition}

We use a metric describing distances in trees as a baseline to compare metric spaces' $\delta$-hyperbolicities.

\begin{definition}[Metric tree]
    A metric space $(X,d)$ is a metric tree if there exists a tree $T=(X,E_T)$ with edges in $E_T$ such that for $u,v\in X$ it holds that $d(u,v)=d_T(u,v)$
\end{definition}
where $d_T(u,v)$ is the length of the shortest path between $u$ and $v$ in the tree $T$. 
 
\begin{figure}
    \centering
    \begin{tikzpicture}
        \node at (0,-1) [circle, draw, fill=black, inner sep=1pt] (z) {};
        \node at (0,-1.7) [circle, draw, fill=black, inner sep=1pt] (a) {};
        \node at (0,-2.4) [circle, draw, fill=black, inner sep=1pt] (b) {};
        \node at (-0.7,-3) [circle, draw, fill=black, inner sep=1pt]  (x) {};
        \node at (0.7,-3) [circle, draw, fill=black, inner sep=1pt] (c) {};
        \node at (1.4,-3.6) [circle, draw, fill=black, inner sep=1pt] (y) {};

        \draw[thick] (z) -- (a);
        \draw[thick] (a) -- (b);
        \draw[thick] (b) -- (x);
        \draw[thick] (b) -- (c);
        \draw[thick] (c) -- (y);

        \draw [decorate,decoration={brace,amplitude=8pt,mirror}] (b) -- (z) node[midway,xshift=10pt,right]{\footnotesize $\langle x,y\rangle_z$};


        \node[left=0.1cm of z] {$z$};
        \node[left=0.1cm of x] {$x$};
        \node[right=0.1cm of y] {$y$};
        

    \end{tikzpicture}
    \caption{Interpretation of the Gromov product in a simple tree.}
    \label{fig:slimTriangle}
\end{figure}


\begin{figure}
    \begin{subfigure}{0.25\textwidth}
        \centering
        \begin{tikzpicture}
            \node at (0,1) [circle, draw, fill=black, inner sep=0.5pt] (z) {};
            \node at (-0.866,-0.5) [circle, draw, fill=black, inner sep=0.5pt]  (x) {};
            \node at (0.866,-0.5) [circle, draw, fill=black, inner sep=0.5pt] (y){};

            
            \draw (z) -- (x);
            \draw (y) -- (x);
            \draw (z) -- (y);

            \clip (0,0) circle (1);
        \end{tikzpicture}
        \caption{In a Euclidean space.}
    \end{subfigure}
    \hfill
    \begin{subfigure}{0.25\textwidth}
        \centering
        \begin{tikzpicture}
            \node at (0,1) [circle, draw, fill=black, inner sep=0.5pt] (z) {};
            \node at (-0.866,-0.5) [circle, draw, fill=black, inner sep=0.5pt] (x) {};
            \node at (0.866,-0.5) [circle, draw, fill=black, inner sep=0.5pt] (y) {};
                        
            \clip (0,0) circle (1);
            \hgline{90}{-30}{black};
            \hgline{-150}{-30}{black};
            \hgline{210}{90}{black};

        \end{tikzpicture}
        \caption{In a hyperbolic space.}
    \end{subfigure}
    \hfill
    \begin{subfigure}{0.25\textwidth}
        \centering
        \begin{tikzpicture}
            \node at (0,1) [circle, draw, fill=black, inner sep=0.5pt] (z) {};
            \node at (0,0.2) [circle, draw, fill=black, inner sep=0.01pt] (b) {};           
            \node at (-0.866,-0.5) [circle, draw, fill=black, inner sep=0.5pt]  (x) {};
            \node at (0.866,-0.5) [circle, draw, fill=black, inner sep=0.5pt] (y){};
            
            \draw (z) -- (b);
            \draw (b) -- (x);
            \draw (b) -- (y);

            \clip (0,0) circle (1);
        \end{tikzpicture}
        \caption{In a tree space.}
    \end{subfigure}
    \caption{Triangles in various metric spaces.}
    \label{fig:triangles}
\end{figure}
One can show that metric trees are $0$-hyperbolic. Let $(X,d)$ be a metric tree, then $\langle x, y\rangle_z$ is the maximum number of edges in $T=(X,E_T)$ between the node $z$ and a common parent for $x$ and $y$ (\Cref{fig:slimTriangle}). It follows that for $x,y,w\in X$ it holds that $\langle x,y\rangle_z \geq \min\{\langle x,w\rangle_z, \langle y, w\rangle_z\}$. Therefore, metric spaces with smaller $\delta$ will be closer to a tree metric. One can show that the Euclidean space is not $\delta$-hyperbolic, whilst hyperbolic spaces are $\log 3$-hyperbolic in the sense of Gromov. This can be noticed by observing triangles in a Euclidean, a hyperbolic and a tree metric (\Cref{fig:triangles}). Indeed we see that with lower $\delta$-hyperbolicity of the metric space, the distance between two vertices increasingly approaches the distance to the ``centre'' of the triangle.

There are important properties such as quasi-isometries between $\delta$-hyperbolic spaces and tree metric spaces. We formalize that any finite set of points in a $\delta$-hyperbolic space can be embedded in a tree metric with bounded distortion~\cite{gromov1987hyperbolic}.

\begin{proposition}[Tree-likeliness of hyperbolic space]
    There is a constant $C_n =\delta \cdot O(n)$ such that for any set of points $x_1, x_2, \dots,x_n$ in a $\delta$-hyperbolic space $(M,d_M)$ can be embedded via $f:M \to T$ into a tree metric $(T,d_T)$ such that:
    \begin{equation*}
        d_M(x_i, x_j)\leq d_T(f(x_i), f(x_j)) \leq d_M(x_i, x_j) + C_n.
    \end{equation*}
\end{proposition}

More specifically, Sarkar~\cite{sarkar2011lowDIstortionDelaunayEmbedding} derived a construction to embed trees in hyperbolic spaces with arbitrarily low-distortion in two dimensions.

\begin{proposition}[Sarkar]
    Any tree $T=(X,E_T)$ can be embedded into $(\mathbb{B}^2, d_\mathbb{B})$ with scale $\zeta = O(1/\varepsilon)$ and worst-case distortion at most $1 + \varepsilon$.
\end{proposition}

These nice quasi-isometries between hyperbolic spaces and tree metric spaces motivate the use of hyperbolic geometry to embed tree-like graphs with low distortion. Since we have these quasi-isometric mappings, hyperbolic spaces can be thought of as some continuous version of trees. We will leverage these properties in the next chapter. 

\subsection{Exponential volume growth}\label{sec:expGrowth}
As we have concluded in the previous section, trees can be thought of as the discrete approximation of some underlying hyperbolic space, where geodesics resemble shortest paths in discrete trees. One can define a notion of volume in trees as the number of nodes contained within some bounded distance (the radius) to the root of the tree. This volume grows exponentially as we increase the radius, as shown in \Cref{fig:radialTree}.

\tikzset{level 1/.style={sibling angle=60,level distance=32mm}}
\tikzset{level 2/.style={sibling angle=35,level distance=16mm}}
\tikzset{level 3/.style={sibling angle=20,level distance=8mm}}
\tikzset{every node/.style={circle, fill=black, inner sep=0pt, color=black}}
\tikzset{edge from parent/.style={segment angle=10,draw}}


\begin{figure}
    \centering
    \begin{tikzpicture}
    [grow cyclic, scale=0.35]
    \node {} 
    child  foreach \A in {1,1,1,1,1,1}{  
    node{} 
        child foreach \B in {2,2,2}{ 
        node {} 
            child foreach \C in {3,3,3}{
            node {} }
        }
    };
    \end{tikzpicture}
    \caption{Volume growth in trees.}
    \label{fig:radialTree}
\end{figure}


In Euclidean space the volume of balls grows polynomially with the radius, making it less suitable to represent exponentially growing graphs like trees. More concretely, because the growth is not as fast as that of the graph, the Euclidean space quickly becomes too crowded, resulting in geodesics intersecting each other.

In contrast, $\delta$-hyperbolic spaces, and in particular hyperbolic spaces, have exponential growth, which yields more room to fit complex hierarchies and makes these spaces ideal candidates to embed trees while accommodating for their exponential volume growth. 





